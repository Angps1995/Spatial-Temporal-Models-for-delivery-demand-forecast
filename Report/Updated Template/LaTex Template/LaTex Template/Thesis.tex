\documentclass[nonblindrev,msom]{informs3} % format for MS submission
\usepackage{amsmath,dcolumn,booktabs}

% Natbib setup for author-year style
\usepackage{natbib}
\usepackage{color}
\usepackage{epstopdf} % Allow importing eps file directly
\bibpunct[, ]{(}{)}{,}{a}{}{,}%
\def\bibfont{\small}%
\def\bibsep{\smallskipamount}%
\def\bibhang{24pt}%
\def\newblock{\ }%
\def\BIBand{and}%

\makeatletter
\renewcommand*\env@matrix[1][c]{\hskip -\arraycolsep
	\let\@ifnextchar\new@ifnextchar
	\array{*\c@MaxMatrixCols #1}}
\makeatother

%% Setup of theorem styles. Outcomment only one.
%% Preferred default is the first option.
\TheoremsNumberedThrough     % Preferred (Theorem 1, Lemma 1, Theorem 2)
%\TheoremsNumberedByChapter  % (Theorem 1.1, Lema 1.1, Theorem 1.2)
\ECRepeatTheorems

%% Setup of the equation numbering system. Outcomment only one.
%% Preferred default is the first option.
\EquationsNumberedThrough    % Default: (1), (2), ...
%\EquationsNumberedBySection % (1.1), (1.2), ...

% For new submissions, leave this number blank.
% For revisions, input the manuscript number assigned by the on-line
% system along with a suffix ".Rx" where x is the revision number.
\MANUSCRIPTNO{}
\newcommand{\vocab}[1]{\textcolor{red}{#1}}

%%%%%%%%%%%%%%%%
\begin{document}
%%%%%%%%%%%%%%%%


% Outcomment only when entries are known. Otherwise leave as is and
%   default values will be used.
%\setcounter{page}{1}
%\VOLUME{00}%
%\NO{0}%
%\MONTH{Xxxxx}% (month or a similar seasonal id)
%\YEAR{0000}% e.g., 2005
%\FIRSTPAGE{000}%
%\LASTPAGE{000}%
%\SHORTYEAR{00}% shortened year (two-digit)
%\ISSUE{0000} %
%\LONGFIRSTPAGE{0001} %
%\DOI{10.1287/xxxx.0000.0000}%

% Author's names for the running heads
% Sample depending on the number of authors;
% \RUNAUTHOR{Jones}
% \RUNAUTHOR{Jones and Wilson}
% \RUNAUTHOR{Jones, Miller, and Wilson}
% \RUNAUTHOR{Jones et al.} % for four or more authors
% Enter authors following the given pattern:
\RUNAUTHOR{Ang Peng Seng}

% Title or shortened title suitable for running heads. Sample:
% \RUNTITLE{Bundling Information Goods of Decreasing Value}
% Enter the (shortened) title:
\RUNTITLE{Classical methods for Spatio-Temporal Forecasting}

% Full title. Sample:
% \TITLE{Bundling Information Goods of Decreasing Value}
% Enter the full title:
\TITLE{NUS Business School Honors Dissertation}

% Block of authors and their affiliations starts here:
% NOTE: Authors with same affiliation, if the order of authors allows,
%   should be entered in ONE field, separated by a comma.
%   \EMAIL field can be repeated if more than one author
\ARTICLEAUTHORS{%
			\AUTHOR{Ang Peng Seng}
			\AFF{National University of Singapore\\
				\EMAIL{e0027055@u.nus.edu}}
	% Enter all authors
}

\ABSTRACT{Many real industry problems involves spatio-temporal demand prediction, which requires predicting demand at a certain time across different locations. Most of the demand data are not continuous variables but instead count variables, like number of orders and number of transactions. Previous research have mainly explored neural network methods for spatio-temporal problems. In this paper, we focus on how we can apply classical time series model, using Vector Autoregressive models to exploit spatial correlations as well as introduce a 3-step approach to improve forecast accuracy. }

\newpage

% Fill in data. If unknown, outcomment the field
\KEYWORDS{Spatio-Temporal demand prediction, Time Series}

\maketitle

% Text of your paper here


\section{Introduction}

\section{Literature Review}

\cite{thomas2009fuel} study...It has been found...\citep{Zhang_SOCPEVModel_2016}

\section{Model}

\section{Analysis}

\section{Computational Results}


\section{Conclusion}

In this paper...



% Appendix here
% Options are (1) APPENDIX (with or without general title) or
%             (2) APPENDICES (if it has more than one unrelated sections)
% Outcomment the appropriate case if necessary
%
% \begin{APPENDIX}{<Title of the Appendix>}
% \end{APPENDIX}
%
%   or
%
% \begin{APPENDICES}
% \section{<Title of Section A>}
% \section{<Title of Section B>}
% etc
% \end{APPENDICES}

%%


% Acknowledgments here
%\ACKNOWLEDGMENT{}


% References here (outcomment the appropriate case)
\bibliographystyle{ormsv080}
%%%\bibliographystyle{abbrvnat}
\bibliography{ref}
% CASE 1: BiBTeX used to constantly update the references
%   (while the paper is being written).
%\bibliographystyle{ormsv080} % outcomment this and next line in Case 1
%\bibliography{<your bib file(s)>} % if more than one, comma separated

% CASE 2: BiBTeX used to generate mypaper.bbl (to be further fine tuned)
%\input{mypaper.bbl} % outcomment this line in Case 2

%If you don't use BiBTex, you can manually itemize references as shown below.


\end{document} 
